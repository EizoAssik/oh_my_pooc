\documentclass[a4paper, 10pt]{article}
\usepackage{CJKutf8}
\usepackage[fleqn]{amsmath}
\usepackage{amstext}
\usepackage{amssymb}
\usepackage{siunitx}
\newcommand{\ud}[1]{\mathrm{d} {#1}}
\begin{document}
\begin{CJK}{UTF8}{gbsn}

\textbf{第七章练习题}
7.11

按照书$P_{261}$, 公式7.56构造求积公式如下

\begin{equation}
	\int_{-1}^{1}f(x)\ud{x} = \frac{5}{9}f(-\frac{\sqrt{15}}{5})+\frac{8}{9}f(0)+\frac{5}{9}f(\frac{\sqrt{15}}{5})
\end{equation}



7.12

按照书$P_{261}$, 公式7.55构造求积公式如下

\begin{equation}
	\int_{-1}^{1}f(x)\ud{x} = f(-\frac{1}{\sqrt{3}})+f(\frac{1}{\sqrt{3}})
\end{equation}


% 构造Gauss-Chebyshev求积公式如下

% \begin{align*}
	% \int^{1}_{-1}f(x)\ud{x} & = \frac{\pi}{2}\sum_{k=1}^{2}{g(cos(\frac{2k-1}{4}))} \\
	                        % & = \frac{\pi}{2}[g(\cos\frac{\pi}{4})+g(\cos\frac{3\pi}{4})] \\
	                        % & = \frac{\pi}{2}[g(\frac{\sqrt{2}}{2})+g(-\frac{\sqrt{2}}{2})]
% \end{align*}

其中

\begin{equation*}
	% g(x) = \sqrt{1+2x}\sqrt{1-x^2}
	f(x) = \sqrt{1+2x}
\end{equation*}

作变量代换, 有

\begin{align*}
	\int_{0}^{1}f(x)\ud(x) & = \int_{-1}^{1}f(\frac{1+t}{2})\ud{t} \\
						   & = \int_{-1}^{1}\sqrt{2+t}\ud{t} \\
						   & = \sqrt{2-\frac{1}{\sqrt{3}}} + \sqrt{2+\frac{1}{\sqrt{3}}}
						   & = 
\end{align*}



\textbf{第八章练习题}

8.1\\
\begin{tabular}{|c | c c c|}
\hline
x    & 1.0    & 2.0    & 3.0 \\
\hline
f(x) & 0.2500 & 0.2268 & 0.2066 \\
\hline
\end{tabular}

构造二次插值多项式$P_2(x)$作为$f(x)$的近似

\begin{equation*}
	P_{2}(x) = \frac{(x-x_{1})(x-x_{2})}{2h^2}f(x_0) - \frac{(x-x_0)(x-x_2)}{h^2} + \frac{(x-x_0)(x-x_1)}{2h^2}
\end{equation*}

代入表格,数据,得

\begin{equation*}
	P_2(x) = \frac{1}{0.1^2}(\frac{0.2500(x-1.1)(x-1.2)}{2}-\frac{0.2268(x-1.0)(x-1.2)}{1}+\frac{0.2066(x-1.0)(x-1.1)}{2})
\end{equation*}

即

\begin{equation*}
	P_2(x) = 0.001250(x-1.1)(x-1.2) - 0.002268(x-1.0)(x-1.2) + 0.001033(x-1.0)(x-1.1)
\end{equation*}

化简有

\begin{equation}
	P_2(x) = \frac{150\,x^2-547\,x+647}{10}
\end{equation}

从而

\begin{equation}
	f'(0.6) = -3.6700405 \times 10^{-5}\,{\it del}\left(0.6\right)
\end{equation}

利用课本$P_{275}$公式8.10

\begin{align*}
	2hf'(1.0) & = -0.0494 \\
	2hf'(1.1) & = -0.0433 \\
	2hf'(1.2) & = -0.0373 \\
\end{align*}

从而

\begin{align*}
	hf'(1.0) & = -0.247 \\
	hf'(1.1) & = -0.216 \\
	hf'(1.2) & = -0.186 \\
\end{align*}

8.2
利用下表, 求$x=0.6$处的导数.\\

\begin{tabular}{c| c c c c c}
\hline
x    & 0.4       & 0.5       & 0.6       & 0.7       & 0.8 \\
\hline
f(x) & 1.5836494 & 1.7974426 & 2.0442376 & 2.3275054 & 2.6510818 \\
\hline
\end{tabular}

利用课本P275公式8.14, 有 \\

\begin{align*}
	12hf'(0.4) & =  2.380297 \\
	12hf'(0.5) & =  2.756952 \\
	12hf'(0.6) & =  3.173070 \\
	12hf'(0.7) & =  3.633028 \\
	12hf'(0.8) & = 20.297579 \\
\end{align*}

从而 \\

\begin{align*}
	f'(0.4) & =  1.983580 \\
	f'(0.5) & =  2.297460 \\
	f'(0.6) & =  2.644225 \\
	f'(0.7) & =  3.027523 \\
	f'(0.8) & = 16.914649 \\
\end{align*}

\textbf{第九章练习题}

9.1
在区间[0, 1]上使用欧拉法解下列初值问题, 取步长h=0.1, 保留到小数点后4位.

\begin{equation*}
	\begin{cases}
		y' = \sin{x}+e^{-x} \\
		y(0) = 0
	\end{cases}
\end{equation*}
\begin{tabular}{c|c c c c c c c c c c c}
\hline
x & 0 & 0.1 & 0.2 & 0.3 & 0.4 & 0.5 & 0.6 & 0.7 & 0.8 & 0.9 & 1.0 \\
\hline
y & 0 & 0.1000 & 0.2005 & 0.3022 & 0.4058 & 0.5118 & 0.6204 & 0.7318 & 0.8458 & 0.9625 & 1.0815 \\
\hline
\end{tabular}
\begin{equation*}
	\begin{cases}
		y' = -y
		y(0) = 2
	\end{cases}
\end{equation*}
\begin{tabular}{c|c c c c c c c c c c c}
\hline
x & 0 & 0.1 & 0.2 & 0.3 & 0.4 & 0.5 & 0.6 & 0.7 & 0.8 & 0.9 & 1.0 \\
\hline
y & 2.0000 & 1.8000 & 1.6200 & 1.4580 & 1.3122 & 1.1810 & 1.0629 & 0.9566 & 0.8609 & 0.7748 & 0.6974 \\
\hline
\end{tabular}

9.2
在区间[0, 1]上用欧拉方法, 改进的欧拉方法和梯形法解初值问题, 取步长为h=0.1, 精确到小数点后4位, 并比较三种算法结果的误差. \\

9.3
用四阶RungeKuttaEuler法求解初值问题, h=0.1精确到小数点后4位. \\
\begin{equation*}
	\begin{cases}
		y' = y^{2}e^{-x} \\
		y(1) = 1, x\in[1,2]
	\end{cases}
\end{equation*}

% [x,y]=RungeKuttaEuler(@(x,y) y^2*exp(-x), 1, 2, 1, 0.1)

\begin{tabular}{c|c c c c c c c c c c c}
\hline
 x & 1.0 & 1.1 & 1.2 & 1.3 & 1.4 & 1.5 & 1.6 & 1.7 & 1.8 & 1.9 & 2.0 \\
 \hline
 y & 1.0000 & 1.0363 & 1.0714 & 1.1054 & 1.1380 & 1.1692 & 1.1990 & 1.2273 & 1.2540 & 1.2793 & 1.3030 \\
 \hline
\end{tabular}

% [x,y]=RungeKuttaEuler(@(x,y) x^2+x^3*y, 1,2, 1,0.1)

\begin{tabular}{c|c c c c c c c c c c c}
\hline
 x & 1.0 & 1.1 & 1.2 & 1.3 & 1.4 & 1.5 & 1.6 & 1.7 & 1.8 & 1.9 & 2.0 \\
\hline
y & 1.0000 & 1.2401 & 1.5873 & 2.1032 & 2.8979 & 4.1785 & 6.3577 & 10.3105 & 18.0306 & 34.4383 & 72.8124 \\
\hline
\end{tabular}

9.6
用欧拉方法和预估-校正方法求解初值问题
\begin{equation}
	\begin{cases}
		y' = x+y \\
		y(0) = 0, x\in[0,1]
	\end{cases}
\end{equation}

h=0.1, 精确到小数点后5位, 并与精确解$y=-x-1+2e^x$相比较

\begin{tabular}{c|c c c c c c c c c c c}
\hline
x & 0 & 0.1 & 0.2 & 0.3 & 0.4 & 0.5 & 0.6 & 0.7 & 0.8 & 0.9 & 1.0 \\
\hline
$y_{Euler}$ & 0 & 0 & 0.01 & 0.03100 & 0.06410 & 0.11051 & 0.17156 & 0.24872 & 0.34359 & 0.45795 & 0.59374 \\
$y_{}$ & oiv \\
\hline
精确值 & 1.00000 & 1.11034 & 1.24281 & 1.39972 & 1.58365 & 1.79744 & 2.04424 & 2.32751 & 2.65108 & 3.01921 & 3.43656 \\
\hline

\end{tabular}
\end{CJK}
\end{document}
